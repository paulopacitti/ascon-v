% Exemplo de relatório técnico do IC

% Criado por P.J.de Rezende antes do Alvorecer da História.
% Modificado em 97-06-15 e 01-02-26 por J.Stolfi.
% modificado em 2003-06-07 21:12:18 por stolfi
% modificado em 2008-10-01 por cll
% modificado em 2010-03-16 17:56:58 por stolfi
% modificado em 2012-09-25 para ajustar o pacote UTF8. Contribuicao de Rogerio Cardoso
% \def\lastedit{2015-03-18 00:52:20 by bit}

\nonstopmode % PARA RODAR LATEX EM BATCH MODE
\documentclass[11pt,twoside]{article}

\usepackage{techrep-ic}

%%% SE USAR INGLÊS, TROQUE AS ATIVAÇÕES DOS DOIS COMANDOS A SEGUIR:
% \usepackage[brazil]{babel}
\usepackage[english]{babel}

%%% SE USAR CODIFICAÇÃO LATIN1 OU UTF-8, ATIVE UM DOS DOIS COMANDOS A
%%% SEGUIR:
%%\usepackage[latin1]{inputenc}
\usepackage[utf8]{inputenc}

%%% Para obter o tamanho de texto recomendado:
\usepackage[margin=1in]{geometry}

\begin{document}

%%% PÁGINA DE CAPA %%%%%%%%%%%%%%%%%%%%%%%%%%%%%%%%%%%%%%%%%%%%%%%
% 
% Número do relatório
\TRNumber{14} % Dois dígitos

% DATA DE PUBLICAÇÃO (PARA A CAPA)
%
\TRYear{23} % Dois dígitos
\TRMonth{11} % Numérico, 01-12

% LISTA DE AUTORES PARA CAPA (sem afiliações).
\TRAuthor{Paulo Pacitti \and Julio López}

% TÍTULO PARA A CAPA (use \\ para forçar quebras de linha).
\TRTitle{Exploring ASCON\\ on RISC-V}

\TRMakeCover

%%%%%%%%%%%%%%%%%%%%%%%%%%%%%%%%%%%%%%%%%%%%%%%%%%%%%%%%%%%%%%%%%%%%%%
% O que segue é apenas uma sugestão - sinta-se à vontade para
% usar seu formato predileto, desde que as margens tenham pelo
% menos 25mm nos quatro lados, e o tamanho do fonte seja pelo menos
% 11pt. Certifique-se também de que o título e lista de autores
% estão reproduzidos na íntegra na página 1, a primeira depois da
% página de capa.
%%%%%%%%%%%%%%%%%%%%%%%%%%%%%%%%%%%%%%%%%%%%%%%%%%%%%%%%%%%%%%%%%%%%%%

%%%%%%%%%%%%%%%%%%%%%%%%%%%%%%%%%%%%%%%%%%%%%%%%%%%%%%%%%%%%%%%%%%%%%%
% Nomes de autores ABREVIADOS e titulo ABREVIADO,
% para cabeçalhos em cada página.
%
\markboth{Pacitti, López}{Cervejas Brasileiras}
\pagestyle{myheadings}
\thispagestyle{empty}

%%%%%%%%%%%%%%%%%%%%%%%%%%%%%%%%%%%%%%%%%%%%%%%%%%%%%%%%%%%%%%%%%%%%%%
% TÍTULO e NOMES DOS AUTORES, completos, para a página 1.
% Use "\\" para quebrar linhas, "\and" para separar autores.
%
\title{Exploring ASCON on RISC-V}

\author{Paulo Pacitti\thanks{Computer Engineering Undergraduate, Institute of Computing, UNICAMP. \texttt{p185447@dac.unicamp.br}} \and
Julio  López\thanks{Associate Professor, Institute of Computing, UNICAMP. \texttt{jlopez@ic.unicamp.br}}}
\date{}
\maketitle

%%%%%%%%%%%%%%%%%%%%%%%%%%%%%%%%%%%%%%%%%%%%%%%%%%%%%%%%%%%%%%%%%%%%%%

\begin{abstract} 
  Sed quis lorem magna. Sed sit amet ullamcorper massa, sit amet placerat lectus. 
  Suspendisse pulvinar ipsum sed enim commodo, ac malesuada lectus finibus. Aliquam eu 
  eros eleifend, interdum nisi faucibus, viverra sapien. Vivamus lobortis a lectus eu rutrum. 
  Quisque in est sit amet libero sollicitudin ornare a sed ipsum. Suspendisse potenti. 
  Aliquam sit amet nisi sed nulla tincidunt imperdiet. Pellentesque elementum lacus eget
  dolor gravida lobortis. Sed placerat lacinia nisi, sed varius turpis facilisis ac. 
\end{abstract}

\section{Introduction}

  Sed quis lorem magna. Sed sit amet ullamcorper massa, sit amet placerat lectus. 
  Suspendisse pulvinar ipsum sed enim commodo, ac malesuada lectus finibus. Aliquam eu 
  eros eleifend, interdum nisi faucibus, viverra sapien. Vivamus lobortis a lectus eu rutrum. 
  Quisque in est sit amet libero sollicitudin ornare a sed ipsum. Suspendisse potenti. 
  Aliquam sit amet nisi sed nulla tincidunt imperdiet. Pellentesque elementum lacus eget
  dolor gravida lobortis. Sed placerat lacinia nisi, sed varius turpis facilisis ac. 


\section{Background}

\subsection{Ascon}

  Ascon is a family of algorithms for lighweight cryptography, designed to be used in constrained environments,
  like embedding computing. Designed by cryptographers from Graz University of Technology, Infineon Technologies, 
  Intel Labs, and Radboud University, Ascon has been selected as the new standard for lightweight cryptography in
  the new NIST Lightweight Cryptography competition (2019–2023).
\subsection{RISC-V}

\section{Implementation}

  The device used for this research is the MangoPi MQ-Pro, a SBC powered with a Allwinner D1 chip and 1GB DDR3 of RAM, with
  Wi-Fi, Bluetooth and HDMI video output. The Allwinner D1 chip contains a T-Head Xuantie C906 core, a RISC-V 64-bit 1GHz CPU
  supporting RV64GC ISA. The board runs Ubuntu Server 23.04, running the latest Linux kernel.

\section{Results}

\section{Discussion}


\section{Conclusions}

 

\begin{thebibliography}{99}

\bibitem{AHU} A. V. Aho, J. E. Hopcroft and J.  D.  Ullman, {\it The
Design and Analysis of Computer Algorithms,} Addison-Wesley (1901).

\bibitem{KNU} D. E. Knuth and L. Lamport, {\it A structural analysis
of the role of gnus and gnats in the post-modernistic, crypto-existential 
Weltanschauung of neo-liberal Tibeto-Vietnamese leaf blower operators 
as manifest in the sexual symbology of the Los Angeles Phone Directory}.
Journal of Gnu Technology, {\bf 23} (6), 12--87
(March 1996).

\end{thebibliography}

\end{document}
